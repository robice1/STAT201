% Options for packages loaded elsewhere
\PassOptionsToPackage{unicode}{hyperref}
\PassOptionsToPackage{hyphens}{url}
%
\documentclass[
]{article}
\usepackage{amsmath,amssymb}
\usepackage{lmodern}
\usepackage{iftex}
\ifPDFTeX
  \usepackage[T1]{fontenc}
  \usepackage[utf8]{inputenc}
  \usepackage{textcomp} % provide euro and other symbols
\else % if luatex or xetex
  \usepackage{unicode-math}
  \defaultfontfeatures{Scale=MatchLowercase}
  \defaultfontfeatures[\rmfamily]{Ligatures=TeX,Scale=1}
\fi
% Use upquote if available, for straight quotes in verbatim environments
\IfFileExists{upquote.sty}{\usepackage{upquote}}{}
\IfFileExists{microtype.sty}{% use microtype if available
  \usepackage[]{microtype}
  \UseMicrotypeSet[protrusion]{basicmath} % disable protrusion for tt fonts
}{}
\makeatletter
\@ifundefined{KOMAClassName}{% if non-KOMA class
  \IfFileExists{parskip.sty}{%
    \usepackage{parskip}
  }{% else
    \setlength{\parindent}{0pt}
    \setlength{\parskip}{6pt plus 2pt minus 1pt}}
}{% if KOMA class
  \KOMAoptions{parskip=half}}
\makeatother
\usepackage{xcolor}
\usepackage[margin=1in]{geometry}
\usepackage{color}
\usepackage{fancyvrb}
\newcommand{\VerbBar}{|}
\newcommand{\VERB}{\Verb[commandchars=\\\{\}]}
\DefineVerbatimEnvironment{Highlighting}{Verbatim}{commandchars=\\\{\}}
% Add ',fontsize=\small' for more characters per line
\usepackage{framed}
\definecolor{shadecolor}{RGB}{248,248,248}
\newenvironment{Shaded}{\begin{snugshade}}{\end{snugshade}}
\newcommand{\AlertTok}[1]{\textcolor[rgb]{0.94,0.16,0.16}{#1}}
\newcommand{\AnnotationTok}[1]{\textcolor[rgb]{0.56,0.35,0.01}{\textbf{\textit{#1}}}}
\newcommand{\AttributeTok}[1]{\textcolor[rgb]{0.77,0.63,0.00}{#1}}
\newcommand{\BaseNTok}[1]{\textcolor[rgb]{0.00,0.00,0.81}{#1}}
\newcommand{\BuiltInTok}[1]{#1}
\newcommand{\CharTok}[1]{\textcolor[rgb]{0.31,0.60,0.02}{#1}}
\newcommand{\CommentTok}[1]{\textcolor[rgb]{0.56,0.35,0.01}{\textit{#1}}}
\newcommand{\CommentVarTok}[1]{\textcolor[rgb]{0.56,0.35,0.01}{\textbf{\textit{#1}}}}
\newcommand{\ConstantTok}[1]{\textcolor[rgb]{0.00,0.00,0.00}{#1}}
\newcommand{\ControlFlowTok}[1]{\textcolor[rgb]{0.13,0.29,0.53}{\textbf{#1}}}
\newcommand{\DataTypeTok}[1]{\textcolor[rgb]{0.13,0.29,0.53}{#1}}
\newcommand{\DecValTok}[1]{\textcolor[rgb]{0.00,0.00,0.81}{#1}}
\newcommand{\DocumentationTok}[1]{\textcolor[rgb]{0.56,0.35,0.01}{\textbf{\textit{#1}}}}
\newcommand{\ErrorTok}[1]{\textcolor[rgb]{0.64,0.00,0.00}{\textbf{#1}}}
\newcommand{\ExtensionTok}[1]{#1}
\newcommand{\FloatTok}[1]{\textcolor[rgb]{0.00,0.00,0.81}{#1}}
\newcommand{\FunctionTok}[1]{\textcolor[rgb]{0.00,0.00,0.00}{#1}}
\newcommand{\ImportTok}[1]{#1}
\newcommand{\InformationTok}[1]{\textcolor[rgb]{0.56,0.35,0.01}{\textbf{\textit{#1}}}}
\newcommand{\KeywordTok}[1]{\textcolor[rgb]{0.13,0.29,0.53}{\textbf{#1}}}
\newcommand{\NormalTok}[1]{#1}
\newcommand{\OperatorTok}[1]{\textcolor[rgb]{0.81,0.36,0.00}{\textbf{#1}}}
\newcommand{\OtherTok}[1]{\textcolor[rgb]{0.56,0.35,0.01}{#1}}
\newcommand{\PreprocessorTok}[1]{\textcolor[rgb]{0.56,0.35,0.01}{\textit{#1}}}
\newcommand{\RegionMarkerTok}[1]{#1}
\newcommand{\SpecialCharTok}[1]{\textcolor[rgb]{0.00,0.00,0.00}{#1}}
\newcommand{\SpecialStringTok}[1]{\textcolor[rgb]{0.31,0.60,0.02}{#1}}
\newcommand{\StringTok}[1]{\textcolor[rgb]{0.31,0.60,0.02}{#1}}
\newcommand{\VariableTok}[1]{\textcolor[rgb]{0.00,0.00,0.00}{#1}}
\newcommand{\VerbatimStringTok}[1]{\textcolor[rgb]{0.31,0.60,0.02}{#1}}
\newcommand{\WarningTok}[1]{\textcolor[rgb]{0.56,0.35,0.01}{\textbf{\textit{#1}}}}
\usepackage{graphicx}
\makeatletter
\def\maxwidth{\ifdim\Gin@nat@width>\linewidth\linewidth\else\Gin@nat@width\fi}
\def\maxheight{\ifdim\Gin@nat@height>\textheight\textheight\else\Gin@nat@height\fi}
\makeatother
% Scale images if necessary, so that they will not overflow the page
% margins by default, and it is still possible to overwrite the defaults
% using explicit options in \includegraphics[width, height, ...]{}
\setkeys{Gin}{width=\maxwidth,height=\maxheight,keepaspectratio}
% Set default figure placement to htbp
\makeatletter
\def\fps@figure{htbp}
\makeatother
\setlength{\emergencystretch}{3em} % prevent overfull lines
\providecommand{\tightlist}{%
  \setlength{\itemsep}{0pt}\setlength{\parskip}{0pt}}
\setcounter{secnumdepth}{-\maxdimen} % remove section numbering
\ifLuaTeX
  \usepackage{selnolig}  % disable illegal ligatures
\fi
\IfFileExists{bookmark.sty}{\usepackage{bookmark}}{\usepackage{hyperref}}
\IfFileExists{xurl.sty}{\usepackage{xurl}}{} % add URL line breaks if available
\urlstyle{same} % disable monospaced font for URLs
\hypersetup{
  pdftitle={Assignment 5},
  hidelinks,
  pdfcreator={LaTeX via pandoc}}

\title{Assignment 5}
\author{}
\date{\vspace{-2.5em}2023-05-11}

\begin{document}
\maketitle

Question 1: An experiment was performed to investigate the effect of
three fertilizers on the size of tomatoes. Six plants were treated with
one of the three fertilizers. At the end of the experiment three
randomly chosen tomatoes were collected from each plant and weighed.

\begin{enumerate}
\def\labelenumi{\alph{enumi}.}
\tightlist
\item
  Read in the data and use str function to see the structure of the
  data. The variables should be: Weight, Fertilizer, Plant.
\end{enumerate}

Using the code below, we can read in our data and confirm that the
structure of the data is correct.

\begin{Shaded}
\begin{Highlighting}[]
\NormalTok{tomatoes\_df }\OtherTok{\textless{}{-}} \FunctionTok{read.csv}\NormalTok{(}\StringTok{"/Users/robbi/Downloads/tomatoes.csv"}\NormalTok{)}
\FunctionTok{str}\NormalTok{(tomatoes\_df)}
\end{Highlighting}
\end{Shaded}

\begin{verbatim}
## 'data.frame':    18 obs. of  4 variables:
##  $ X         : int  1 2 3 4 5 6 7 8 9 10 ...
##  $ Weight    : num  11.29 11.08 11.1 7.37 6.55 ...
##  $ Fertilizer: chr  "0" "0" "0" "0" ...
##  $ Plant     : int  1 1 1 2 2 2 3 3 3 4 ...
\end{verbatim}

We can see that the structure of the data is correct, as the variables
`Weight', `Fertilizer' and `Plant' are all present in our data frame.

\begin{enumerate}
\def\labelenumi{\alph{enumi}.}
\setcounter{enumi}{1}
\tightlist
\item
  Explore the data using the graphs below. What do these graphs tell
  you?
\end{enumerate}

\begin{Shaded}
\begin{Highlighting}[]
\FunctionTok{library}\NormalTok{(ggplot2)}
\FunctionTok{ggplot}\NormalTok{(tomatoes\_df, }\FunctionTok{aes}\NormalTok{(}\AttributeTok{x=}\NormalTok{Fertilizer, }\AttributeTok{y=}\NormalTok{Weight, }\AttributeTok{color=}\NormalTok{Fertilizer)) }\SpecialCharTok{+} \FunctionTok{geom\_point}\NormalTok{()}
\end{Highlighting}
\end{Shaded}

\includegraphics{Assignment-5_files/figure-latex/unnamed-chunk-2-1.pdf}

\begin{Shaded}
\begin{Highlighting}[]
\FunctionTok{ggplot}\NormalTok{(tomatoes\_df, }\FunctionTok{aes}\NormalTok{(}\AttributeTok{x=}\NormalTok{Fertilizer, }\AttributeTok{y=}\NormalTok{Weight, }\AttributeTok{color=}\NormalTok{Fertilizer)) }\SpecialCharTok{+} \FunctionTok{geom\_point}\NormalTok{() }\SpecialCharTok{+}
\FunctionTok{facet\_grid}\NormalTok{(}\SpecialCharTok{\textasciitilde{}}\NormalTok{Plant)}
\end{Highlighting}
\end{Shaded}

\includegraphics{Assignment-5_files/figure-latex/unnamed-chunk-2-2.pdf}
The first graph is a scatterplot that shows the distribution of weights
of tomatoes grown in different fertilizers. The second graph adds the
facet\_grid function for the plant variable so we can see the
distribution of weights with each of the plants being separated to their
own scatterplot

\begin{enumerate}
\def\labelenumi{\alph{enumi}.}
\setcounter{enumi}{2}
\tightlist
\item
  Formulate the research question.
\end{enumerate}

The research question for this dataset could be `Does the type of
fertilizer that a tomato plant grows in affect the weight of the
tomatoes?'

\begin{enumerate}
\def\labelenumi{\alph{enumi}.}
\setcounter{enumi}{3}
\tightlist
\item
  A linear mixed-effects model is to be used to answer the research
  question. Which of the variables are response variable, explanatory
  component (fixed effects), and structural component (random effects)?
\end{enumerate}

The response variable is the weight of the tomato, as it is what is
being measured to answer our question. The explanatory component is the
type of fertilizer that the tomatoes are being grown in as it what we
are observing the weights of the tomatoes in. The structural component
is each tomato plant as they represent a group of tomatoes within the
data.

\begin{enumerate}
\def\labelenumi{\alph{enumi}.}
\setcounter{enumi}{4}
\tightlist
\item
  Convert all categorical variables to factors, fit two linear mixed
  models using lmer function from lme4 package. Use anova function to
  test for the significance of the fixed effects. Are the fixed effects
  significant? Use the following code replacing V, response,
  fixed\_effect, and random\_effect with appropriate variables:
\end{enumerate}

In the code below, V and fixed\_effect was replaced with Fertilizer,
response was replaced with Weight and random\_effect was replaced with
Plant.

\begin{Shaded}
\begin{Highlighting}[]
\CommentTok{\#install.packages("lme4")}
\FunctionTok{library}\NormalTok{(lme4)}
\end{Highlighting}
\end{Shaded}

\begin{verbatim}
## Warning: package 'lme4' was built under R version 4.2.3
\end{verbatim}

\begin{verbatim}
## Loading required package: Matrix
\end{verbatim}

\begin{Shaded}
\begin{Highlighting}[]
\NormalTok{tomatoes\_df}\SpecialCharTok{$}\NormalTok{Fertilizer }\OtherTok{\textless{}{-}} \FunctionTok{as.factor}\NormalTok{(tomatoes\_df}\SpecialCharTok{$}\NormalTok{Fertilizer)}
\NormalTok{tomatoes\_lmm0 }\OtherTok{\textless{}{-}} \FunctionTok{lmer}\NormalTok{(Weight }\SpecialCharTok{\textasciitilde{}}\NormalTok{ (}\DecValTok{1}\SpecialCharTok{|}\NormalTok{Plant), }\AttributeTok{data=}\NormalTok{tomatoes\_df)}
\NormalTok{tomatoes\_lmm1 }\OtherTok{\textless{}{-}} \FunctionTok{lmer}\NormalTok{(Weight }\SpecialCharTok{\textasciitilde{}}\NormalTok{ Fertilizer }\SpecialCharTok{+}\NormalTok{ (}\DecValTok{1}\SpecialCharTok{|}\NormalTok{Plant), }\AttributeTok{data=}\NormalTok{tomatoes\_df)}
\FunctionTok{anova}\NormalTok{(tomatoes\_lmm0,tomatoes\_lmm1)}
\end{Highlighting}
\end{Shaded}

\begin{verbatim}
## refitting model(s) with ML (instead of REML)
\end{verbatim}

\begin{verbatim}
## Data: tomatoes_df
## Models:
## tomatoes_lmm0: Weight ~ (1 | Plant)
## tomatoes_lmm1: Weight ~ Fertilizer + (1 | Plant)
##               npar    AIC    BIC  logLik deviance  Chisq Df Pr(>Chisq)  
## tomatoes_lmm0    3 61.537 64.208 -27.769   55.537                       
## tomatoes_lmm1    5 57.394 61.846 -23.697   47.394 8.1434  2    0.01705 *
## ---
## Signif. codes:  0 '***' 0.001 '**' 0.01 '*' 0.05 '.' 0.1 ' ' 1
\end{verbatim}

The output for the anova function can be used to test the significance
of the fixed effects. The p-value for the fixed effect of fertilizer in
the second model is 0.01705, which is less than 0.05. This shows that
the type of fertilizer used most likely have a significant effect on the
weight of the tomatoes.

\begin{enumerate}
\def\labelenumi{\alph{enumi}.}
\setcounter{enumi}{5}
\tightlist
\item
  Use lmerTest and ranova function from lmerTest library to test for the
  significance of the random effects. Are the random effects
  significant?
\end{enumerate}

Using the code below we can test the signifance of the random effects
(the plant variable).

\begin{Shaded}
\begin{Highlighting}[]
\CommentTok{\#install.packages("lmerTest")}
\FunctionTok{library}\NormalTok{(lmerTest)}
\end{Highlighting}
\end{Shaded}

\begin{verbatim}
## Warning: package 'lmerTest' was built under R version 4.2.3
\end{verbatim}

\begin{verbatim}
## 
## Attaching package: 'lmerTest'
\end{verbatim}

\begin{verbatim}
## The following object is masked from 'package:lme4':
## 
##     lmer
\end{verbatim}

\begin{verbatim}
## The following object is masked from 'package:stats':
## 
##     step
\end{verbatim}

\begin{Shaded}
\begin{Highlighting}[]
\FunctionTok{ranova}\NormalTok{(tomatoes\_lmm1)}
\end{Highlighting}
\end{Shaded}

\begin{verbatim}
## ANOVA-like table for random-effects: Single term deletions
## 
## Model:
## Weight ~ Fertilizer + (1 | Plant)
##             npar  logLik    AIC    LRT Df Pr(>Chisq)    
## <none>         5 -20.579 51.158                         
## (1 | Plant)    4 -30.827 69.654 20.496  1  5.975e-06 ***
## ---
## Signif. codes:  0 '***' 0.001 '**' 0.01 '*' 0.05 '.' 0.1 ' ' 1
\end{verbatim}

We can see that the p-value for plant is much less than 0.05 (5.975e-6).
This indicates that the random effect of plant is significant to our
research question.

\begin{enumerate}
\def\labelenumi{\alph{enumi}.}
\setcounter{enumi}{6}
\tightlist
\item
  Based on e and f, answer the research question.
\end{enumerate}

From our answers to e and f, we can answer the research question with
the following statement: Both the explanatory component of the type of
fertilizer and the random effect of the plant the tomato is apart of
have an effect on the weights of tomatoes.

\begin{enumerate}
\def\labelenumi{\alph{enumi}.}
\setcounter{enumi}{7}
\tightlist
\item
  Check the assumptions of the model using the code below. Are the
  assumptions met?
\end{enumerate}

To check the assumptions of the model, we create a Q-Q plot of the
residuals and a fitted versus residuals plot.

\begin{Shaded}
\begin{Highlighting}[]
\FunctionTok{qqnorm}\NormalTok{(}\FunctionTok{resid}\NormalTok{(tomatoes\_lmm1))}
\FunctionTok{qqline}\NormalTok{(}\FunctionTok{resid}\NormalTok{(tomatoes\_lmm1), }\AttributeTok{col=}\DecValTok{2}\NormalTok{)}
\end{Highlighting}
\end{Shaded}

\includegraphics{Assignment-5_files/figure-latex/unnamed-chunk-5-1.pdf}

\begin{Shaded}
\begin{Highlighting}[]
\FunctionTok{plot}\NormalTok{(tomatoes\_lmm1)}
\end{Highlighting}
\end{Shaded}

\includegraphics{Assignment-5_files/figure-latex/unnamed-chunk-5-2.pdf}
The Q-Q plot shows that the residuals approximately follow a normal
distribution. The residuals versus fitted plot has no clear pattern or
trend across it. This shows that our assumptions of homoscedasticity,
linearity and normality are correct.

Question 2

To investigate whether darker frogs have lower skin microbial diversity,
frogs of five different species were captured, their darkness was
measured and skin microbiome tests were performed. Based on the tests
the diversity score was recorded.

\begin{enumerate}
\def\labelenumi{\alph{enumi}.}
\tightlist
\item
  Read in the data and see the structure of the data using str function.
  The variable names are: Diversity, Darkness, and Species.
\end{enumerate}

Using the code below, we can read in the data and see its structure to
verify that it is correct.

\begin{Shaded}
\begin{Highlighting}[]
\NormalTok{frogs\_df }\OtherTok{\textless{}{-}} \FunctionTok{read.csv}\NormalTok{(}\StringTok{"/Users/robbi/Downloads/frogs.csv"}\NormalTok{)}
\FunctionTok{str}\NormalTok{(frogs\_df)}
\end{Highlighting}
\end{Shaded}

\begin{verbatim}
## 'data.frame':    50 obs. of  4 variables:
##  $ X        : int  1 2 3 4 5 6 7 8 9 10 ...
##  $ Diversity: num  2.53 2.75 3.16 2.17 3.07 ...
##  $ Darkness : num  0.987 0.995 1.586 1.879 1.03 ...
##  $ Species  : int  1 1 1 1 1 1 1 1 1 1 ...
\end{verbatim}

We see that the correct variables of Diversity, Darkness, Species are
apart of the data frame.

\begin{enumerate}
\def\labelenumi{\alph{enumi}.}
\setcounter{enumi}{1}
\tightlist
\item
  Formulate the research question.
\end{enumerate}

The research question for this can be: Are there any relationships
between frog darkness and skin microbial diversity?

\begin{enumerate}
\def\labelenumi{\alph{enumi}.}
\setcounter{enumi}{2}
\tightlist
\item
  Convert any categorical variables to factors. Plot the data using the
  code below. Based on the plots, explain why you would use a linear
  mixed model for this analysis.
\end{enumerate}

\begin{Shaded}
\begin{Highlighting}[]
\NormalTok{frogs\_df}\SpecialCharTok{$}\NormalTok{Species }\OtherTok{\textless{}{-}} \FunctionTok{as.factor}\NormalTok{(frogs\_df}\SpecialCharTok{$}\NormalTok{Species)}
\FunctionTok{library}\NormalTok{(ggplot2)}
\FunctionTok{ggplot}\NormalTok{(frogs\_df, }\FunctionTok{aes}\NormalTok{(}\AttributeTok{x =}\NormalTok{ Darkness, }\AttributeTok{y =}\NormalTok{ Diversity)) }\SpecialCharTok{+}
\FunctionTok{geom\_point}\NormalTok{()}
\end{Highlighting}
\end{Shaded}

\includegraphics{Assignment-5_files/figure-latex/unnamed-chunk-7-1.pdf}

\begin{Shaded}
\begin{Highlighting}[]
\FunctionTok{ggplot}\NormalTok{(frogs\_df, }\FunctionTok{aes}\NormalTok{(}\AttributeTok{x=}\NormalTok{Species, }\AttributeTok{y=}\NormalTok{Diversity, }\AttributeTok{color=}\NormalTok{Species))}\SpecialCharTok{+}\FunctionTok{geom\_boxplot}\NormalTok{()}
\end{Highlighting}
\end{Shaded}

\includegraphics{Assignment-5_files/figure-latex/unnamed-chunk-7-2.pdf}
From these plots, we can conclude that a linear mixed model for the
analysis is the best. This is because the boxplot shows differences in
diversity for each of the species and the scatterplot shows there is
variation in diversity based on the level of darkness. This indicates
that there is possibly other factors that effect the diversity than just
darkness.

\begin{enumerate}
\def\labelenumi{\alph{enumi}.}
\setcounter{enumi}{3}
\tightlist
\item
  Fit a linear mixed model. Answer the research question stated above
  using anova function to compare models with and without fixed effects.
  Use the code below replacing response, fixed\_effect, and
  random\_effect with appropriate variables:
\end{enumerate}

In the code below, we replaced the response variable with Diversity, the
fixed\_effect variable with Darkness, and the random\_effect variable
with Species.

\begin{Shaded}
\begin{Highlighting}[]
\NormalTok{mlm.frogs\_df }\OtherTok{\textless{}{-}} \FunctionTok{lmer}\NormalTok{(Diversity }\SpecialCharTok{\textasciitilde{}}\NormalTok{ Darkness }\SpecialCharTok{+}\NormalTok{ (}\DecValTok{1}\SpecialCharTok{|}\NormalTok{Species), }\AttributeTok{data =}\NormalTok{ frogs\_df)}
\NormalTok{mlm.frogs\_df0 }\OtherTok{\textless{}{-}} \FunctionTok{lmer}\NormalTok{(Diversity }\SpecialCharTok{\textasciitilde{}} \DecValTok{1} \SpecialCharTok{+}\NormalTok{ (}\DecValTok{1}\SpecialCharTok{|}\NormalTok{Species), }\AttributeTok{data =}\NormalTok{ frogs\_df)}
\FunctionTok{anova}\NormalTok{(mlm.frogs\_df0, mlm.frogs\_df)}
\end{Highlighting}
\end{Shaded}

\begin{verbatim}
## refitting model(s) with ML (instead of REML)
\end{verbatim}

\begin{verbatim}
## Data: frogs_df
## Models:
## mlm.frogs_df0: Diversity ~ 1 + (1 | Species)
## mlm.frogs_df: Diversity ~ Darkness + (1 | Species)
##               npar    AIC     BIC logLik deviance  Chisq Df Pr(>Chisq)    
## mlm.frogs_df0    3 95.919 101.655 -44.96   89.919                         
## mlm.frogs_df     4 73.920  81.568 -32.96   65.920 23.999  1  9.638e-07 ***
## ---
## Signif. codes:  0 '***' 0.001 '**' 0.01 '*' 0.05 '.' 0.1 ' ' 1
\end{verbatim}

\begin{enumerate}
\def\labelenumi{\alph{enumi}.}
\setcounter{enumi}{4}
\tightlist
\item
  Check the assumptions of the model. Use the code below:
\end{enumerate}

\begin{Shaded}
\begin{Highlighting}[]
\FunctionTok{qqnorm}\NormalTok{(}\FunctionTok{resid}\NormalTok{(mlm.frogs\_df))}
\FunctionTok{qqline}\NormalTok{(}\FunctionTok{resid}\NormalTok{(mlm.frogs\_df), }\AttributeTok{col=}\DecValTok{2}\NormalTok{)}
\end{Highlighting}
\end{Shaded}

\includegraphics{Assignment-5_files/figure-latex/unnamed-chunk-9-1.pdf}

\begin{Shaded}
\begin{Highlighting}[]
\FunctionTok{plot}\NormalTok{(mlm.frogs\_df)}
\end{Highlighting}
\end{Shaded}

\includegraphics{Assignment-5_files/figure-latex/unnamed-chunk-9-2.pdf}
We can see from the Normal Q-Q plot that the residuals are roughly
normally distributed. The residuals versus fitted plot shows no clear
trends or patterns across the values. This shows the assumptions of
normality, linearity and homoscedasticity are reasonable assumptions.

\end{document}
